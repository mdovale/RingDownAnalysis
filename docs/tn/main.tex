\documentclass[11pt]{article}

\usepackage[margin=1in]{geometry}
\usepackage{amsmath,amssymb}
\usepackage{siunitx}
\usepackage{hyperref}
\usepackage{graphicx}

\setlength{\parindent}{0pt}
\setlength{\parskip}{0.3\baselineskip}

\title{Frequency and Q-factor estimation of ring-down signals: nonlinear least squares and discrete Fourier transform methods}
\author{Miguel Dovale for the LASSO Team, mdovale@arizona.edu}
\date{}

\begin{document}
\maketitle

\begin{abstract}
We analyze the accuracy of estimating the frequency and quality factor $Q$ of a decaying sinusoid from finite, noisy samples. The signal arises from ring-down measurements of a harmonic oscillator, where the amplitude decays exponentially. Two complementary approaches are presented: nonlinear least squares (NLS) with explicit ring-down model, and frequency-domain methods (DFT followed by peak location with Lorentzian fitting). We derive the Fisher information matrix explicitly for the ring-down case and compare estimator performance against the Cram\'er--Rao lower bound (CRLB). We provide a complete derivation of the CRLB for $Q$, showing that under the standard high-SNR, many-cycle approximation, frequency and decay time are asymptotically uncorrelated after marginalizing nuisance parameters.
\end{abstract}

\section{Problem formulation}

Consider a ring-down measurement of a harmonic oscillator, where the signal amplitude decays exponentially due to energy dissipation. The real-valued signal sampled at times $t_n$ is:
\begin{equation}
x_n = A_0 e^{-t_n/\tau}\cos(2\pi f_0 t_n + \phi_0) + w_n,\qquad n=0,1,\dots,N-1,
\end{equation}
where $f_0$ is the unknown resonance frequency, $A_0$ is the initial amplitude, $\phi_0$ is the initial phase, $\tau$ is the decay time constant, and $w_n$ is additive noise. The decay time constant is related to the quality factor $Q$ by
\begin{equation}
\tau = \frac{Q}{\pi f_0}.
\end{equation}
The sampling times are
\begin{equation}
t_n = nT_s,
\end{equation}
where $T_s$ is the nominal sampling period and $T = NT_s$ is the total observation time.

For white Gaussian noise $w_n \sim \mathcal{N}(0,\sigma^2)$, the instantaneous signal-to-noise ratio decreases exponentially:
\begin{equation}
\rho(t) = \frac{[A_0 e^{-t/\tau}]^2/2}{\sigma^2} = \rho_0 e^{-2t/\tau},
\end{equation}
where $\rho_0 = A_0^2/(2\sigma^2)$ is the initial SNR. Later samples contribute less information due to lower amplitude and SNR.

\section{Fisher information and Cram\'er--Rao lower bound}

\subsection{Explicit Fisher information matrix}

For the ring-down model with $\omega = 2\pi f$ and $t_n = nT_s$, the parameter vector is $\theta = [A_0, \phi, \omega, \tau]^T$, and $w_n \sim \mathcal{N}(0,\sigma^2)$.

The mean function is $\mu_n(\theta) = A_0 e^{-t_n/\tau}\cos(\omega t_n + \phi)$. The Fisher information matrix elements are
\begin{equation}
\mathcal{I}_{ij} = \frac{1}{\sigma^2}\sum_{n=0}^{N-1}
\frac{\partial \mu_n}{\partial \theta_i}\frac{\partial \mu_n}{\partial \theta_j}.
\end{equation}

The partial derivatives are:
\begin{align}
\frac{\partial \mu_n}{\partial A_0} &= e^{-t_n/\tau}\cos(\omega t_n + \phi),\\
\frac{\partial \mu_n}{\partial \phi} &= -A_0 e^{-t_n/\tau}\sin(\omega t_n + \phi),\\
\frac{\partial \mu_n}{\partial \omega} &= -A_0 t_n e^{-t_n/\tau}\sin(\omega t_n + \phi),\\
\frac{\partial \mu_n}{\partial \tau} &= A_0 \frac{t_n}{\tau^2} e^{-t_n/\tau}\cos(\omega t_n + \phi).
\end{align}

For long records ($N \gg 1$, $\omega T \gg 2\pi$), trigonometric cross-terms average to zero, and squared terms average to $1/2$. The Fisher information matrix becomes approximately:
\begin{equation}
\mathcal{I} \approx \frac{1}{\sigma^2}
\begin{bmatrix}
\sum_n e^{-2t_n/\tau} & 0 & 0 & \sum_n \frac{t_n}{\tau^2} e^{-2t_n/\tau}\\
0 & A_0^2 \sum_n e^{-2t_n/\tau} & A_0^2 \sum_n t_n e^{-2t_n/\tau} & 0\\
0 & A_0^2 \sum_n t_n e^{-2t_n/\tau} & A_0^2 \sum_n t_n^2 e^{-2t_n/\tau} & 0\\
\sum_n \frac{t_n}{\tau^2} e^{-2t_n/\tau} & 0 & 0 & A_0^2 \sum_n \frac{t_n^2}{\tau^4} e^{-2t_n/\tau}
\end{bmatrix}.
\label{eq:fisher-matrix}
\end{equation}

Define the weighted sums (with $t_n = nT_s$):
\begin{align}
S_0 &= \sum_{n=0}^{N-1} e^{-2nT_s/\tau}, &
S_1 &= \sum_{n=0}^{N-1} nT_s e^{-2nT_s/\tau}, &
S_2 &= \sum_{n=0}^{N-1} (nT_s)^2 e^{-2nT_s/\tau}.
\end{align}

For $T \gg \tau$, the sums saturate:
\begin{align}
S_0 &\approx \frac{\tau}{2T_s}, &
S_1 &\approx \frac{\tau^2}{4T_s}, &
S_2 &\approx \frac{\tau^3}{4T_s}.
\end{align}

\subsection{Cram\'er--Rao lower bound for frequency}

After marginalizing $A_0$ and $\phi$, the effective Fisher information for frequency is:
\begin{equation}
\mathcal{I}_{\text{eff}}(\omega) = \frac{A_0^2}{\sigma^2}
\left(
S_2 - \frac{S_1^2}{S_0}
\right).
\end{equation}

Thus the CRLB for cyclic frequency $f = \omega/(2\pi)$ is:
\begin{equation}
\boxed{
\sigma_f^2 \ge \frac{1}{(2\pi)^2} \frac{\sigma^2}{A_0^2\left(S_2 - S_1^2/S_0\right)}.
\label{eq:crlb-ringdown}
}
\end{equation}

For $T \ll \tau$: $S_0 \approx N$, $S_1 \approx NT/2$, $S_2 \approx NT^2/3$, yielding
$\sigma_f^2 \ge 12/[ (2\pi)^2 \rho_0 T_s^2 N^3 ]$, the known result for a constant-amplitude sinusoid.

For $T \gg \tau$: $S_2 - S_1^2/S_0 = \frac{\tau^3}{4T_s} - \frac{\tau^4/(16T_s^2)}{\tau/(2T_s)} = \frac{\tau^3}{8T_s}$, so
\begin{equation}
\sigma_f^2 \ge \frac{8T_s}{(2\pi)^2} \frac{\sigma^2}{A_0^2 \tau^3}.
\end{equation}
The bound degrades as $\tau^{-3}$, reflecting that only $\sim \tau$ of observation time is effective.

\subsection{Cram\'er--Rao lower bound for quality factor}

The quality factor is $Q = \pi f \tau = \frac{\omega \tau}{2}$. Since $Q$ is a function of two parameters, its CRLB depends on the covariance of $(\omega, \tau)$ estimates.

\subsubsection{Effective Fisher information for $(\omega, \tau)$}

To obtain the CRLB for $Q$, we must compute the effective $2\times2$ Fisher matrix for $(\omega, \tau)$ after marginalizing the nuisance parameters $(A_0, \phi)$. Partition the full $4\times4$ FIM as
\[
\mathcal{I} = 
\begin{bmatrix}
\mathcal{I}_{AA} & \mathbf{0}^T & \mathcal{I}_{A\omega} & \mathcal{I}_{A\tau} \\
\mathbf{0} & \mathcal{I}_{\phi\phi} & \mathcal{I}_{\phi\omega} & \mathbf{0}^T \\
\mathcal{I}_{\omega A} & \mathcal{I}_{\omega\phi} & \mathcal{I}_{\omega\omega} & \mathbf{0} \\
\mathcal{I}_{\tau A} & \mathbf{0} & \mathbf{0} & \mathcal{I}_{\tau\tau}
\end{bmatrix},
\]
where, using the weighted sums:
\begin{align}
\mathcal{I}_{AA} &= \frac{1}{\sigma^2} S_0, &
\mathcal{I}_{\phi\phi} &= \frac{A_0^2}{\sigma^2} S_0, \\
\mathcal{I}_{A\omega} &= 0, &
\mathcal{I}_{A\tau} &= \frac{1}{\sigma^2} \frac{S_1}{\tau^2}, \\
\mathcal{I}_{\phi\omega} &= \frac{A_0^2}{\sigma^2} S_1, &
\mathcal{I}_{\omega\omega} &= \frac{A_0^2}{\sigma^2} S_2, \\
\mathcal{I}_{\tau\tau} &= \frac{A_0^2}{\sigma^2} \frac{S_2}{\tau^4}, &
\mathcal{I}_{\omega\tau} &= 0.
\end{align}

Let $\mathbf{z} = [\omega, \tau]^T$ be the parameters of interest, and $\mathbf{u} = [A_0, \phi]^T$ the nuisance parameters. The Schur complement gives the effective Fisher information for $\mathbf{z}$:
\[
\mathcal{I}_{\text{eff}}(\mathbf{z}) = 
\begin{bmatrix}
\mathcal{I}_{\omega\omega} & \mathcal{I}_{\omega\tau} \\
\mathcal{I}_{\tau\omega} & \mathcal{I}_{\tau\tau}
\end{bmatrix}
-
\begin{bmatrix}
\mathcal{I}_{\omega A} & \mathcal{I}_{\omega\phi}  \\
\mathcal{I}_{\tau A} & \mathcal{I}_{\tau\phi}
\end{bmatrix}
\begin{bmatrix}
\mathcal{I}_{AA} & 0 \\
0 & \mathcal{I}_{\phi\phi}
\end{bmatrix}^{-1}
\begin{bmatrix}
\mathcal{I}_{A\omega} & \mathcal{I}_{A\tau} \\
\mathcal{I}_{\phi\omega} & \mathcal{I}_{\phi\tau}
\end{bmatrix}.
\]

Since $\mathcal{I}_{\omega A} = \mathcal{I}_{A\omega} = 0$ and $\mathcal{I}_{\tau\phi} = \mathcal{I}_{\phi\tau} = 0$ in the averaged FIM (Eq.~\eqref{eq:fisher-matrix}), the Schur complement yields:
\begin{align}
\mathcal{I}_{\omega\omega}^{\text{eff}} &= \mathcal{I}_{\omega\omega} - \frac{\mathcal{I}_{\omega\phi}^2}{\mathcal{I}_{\phi\phi}}
= \frac{A_0^2}{\sigma^2} \left( S_2 - \frac{S_1^2}{S_0} \right), \\
\mathcal{I}_{\tau\tau}^{\text{eff}} &= \mathcal{I}_{\tau\tau} - \frac{\mathcal{I}_{\tau A}^2}{\mathcal{I}_{AA}}
= \frac{A_0^2}{\sigma^2 \tau^4} \left( S_2 - \frac{S_1^2}{S_0} \right), \\
\mathcal{I}_{\omega\tau}^{\text{eff}} &= \mathcal{I}_{\omega\tau} - \frac{\mathcal{I}_{\omega A} \mathcal{I}_{A\tau}}{\mathcal{I}_{AA}} - \frac{\mathcal{I}_{\omega\phi} \mathcal{I}_{\phi\tau}}{\mathcal{I}_{\phi\phi}} = 0,
\end{align}
since $\mathcal{I}_{\omega\tau} = 0$, $\mathcal{I}_{\omega A} = 0$, and $\mathcal{I}_{\phi\tau} = 0$. Therefore,
\[
\boxed{
\mathcal{I}_{\omega\tau}^{\text{eff}} = 0.
}
\]
Under the standard high-SNR, many-cycle approximation, $\omega$ and $\tau$ are asymptotically uncorrelated even after marginalizing $A_0$ and $\phi$. The effective FIM for $(\omega, \tau)$ remains diagonal.

Thus,
\begin{equation}
\mathcal{I}_{\text{eff}}(\omega,\tau) =
\frac{A_0^2}{\sigma^2}
\begin{bmatrix}
S_2 - S_1^2/S_0 & 0 \\
0 & \frac{1}{\tau^4}\left(S_2 - S_1^2/S_0\right)
\end{bmatrix}
\equiv
\frac{A_0^2}{\sigma^2} \, \Delta S_2 \,
\begin{bmatrix}
1 & 0 \\
0 & \tau^{-4}
\end{bmatrix},
\label{eq:I_eff_wt}
\end{equation}
where $\Delta S_2 \triangleq S_2 - S_1^2/S_0 > 0$.

The covariance matrix for $(\omega, \tau)$ is the inverse:
\[
\operatorname{Cov}(\omega, \tau) \succeq
\frac{\sigma^2}{A_0^2 \Delta S_2}
\begin{bmatrix}
1 & 0 \\
0 & \tau^4
\end{bmatrix}.
\]

\subsubsection{CRLB for $Q$}

Let $Q = g(\omega, \tau) = \omega \tau / 2$. The Jacobian is
\[
\mathbf{J} = \begin{bmatrix}
\partial Q/\partial\omega & \partial Q/\partial\tau
\end{bmatrix}
= \begin{bmatrix}
\tau/2 & \omega/2
\end{bmatrix}.
\]

The CRLB for $Q$ is:
\begin{align}
\sigma_Q^2 &\ge \mathbf{J} \operatorname{Cov}(\omega,\tau) \mathbf{J}^T \nonumber \\
&= \frac{\sigma^2}{A_0^2 \Delta S_2}
\left[
\left(\frac{\tau}{2}\right)^2 \cdot 1 + \left(\frac{\omega}{2}\right)^2 \cdot \tau^4
\right] \nonumber \\
&= \frac{\sigma^2}{A_0^2 \Delta S_2} \cdot \frac{\tau^2}{4} \left( 1 + \omega^2 \tau^2 \right).
\end{align}

Since $Q = \omega \tau / 2$, we have $\omega^2 \tau^2 = 4Q^2$, so
\begin{equation}
\sigma_Q^2 \ge \frac{\sigma^2 \tau^2}{4 A_0^2 \Delta S_2} \left( 1 + 4Q^2 \right).
\label{eq:crlb-Q-final}
\end{equation}

This is the asymptotic CRLB for $Q$.

\subsection{CRLB scaling laws}
\label{sec:crlb-scaling}

The CRLB expressions in Eqs.~\eqref{eq:crlb-ringdown} and \eqref{eq:crlb-Q-final} exhibit distinct scaling behaviors in different regimes. The essential scaling relationships are summarized below.

\subsubsection{Frequency estimation scaling}

For frequency estimation, the CRLB depends on the ratio $T/\tau$:

\begin{itemize}
\item \textbf{Slow-decay regime} ($T \ll \tau$): Using $S_0 \approx N$, $S_1 \approx NT/2$, $S_2 \approx NT^2/3$ in Eq.~\eqref{eq:crlb-ringdown},
  \begin{equation}
  \sigma_f^2 \ge \frac{12}{(2\pi)^2 \rho_0 T_s^2 N^3} = \frac{12}{(2\pi)^2 \rho_0 T^3 f_s},
  \label{eq:scaling-slow-decay}
  \end{equation}
  where $\rho_0 = A_0^2/(2\sigma^2)$ is the initial SNR. This recovers the $T^{-3}$ scaling for constant-amplitude sinusoids.

\item \textbf{Rapid-decay regime} ($T \gg \tau$): Using the saturated sums $S_0 \approx \tau/(2T_s)$, $S_1 \approx \tau^2/(4T_s)$, $S_2 \approx \tau^3/(4T_s)$,
  \begin{equation}
  \sigma_f^2 \ge \frac{8T_s}{(2\pi)^2 \rho_0 \tau^3}.
  \label{eq:scaling-rapid-decay}
  \end{equation}
  The bound becomes independent of $T$ and degrades as $\tau^{-3}$.

\item \textbf{SNR scaling}: In both regimes, $\sigma_f \propto 1/\sqrt{\rho_0}$, as shown in Fig.~\ref{fig:crlb-freq-snr}.
\end{itemize}

The transition between regimes is illustrated in Fig.~\ref{fig:crlb-freq-tau-ratio}.

\subsubsection{Quality factor estimation scaling}

For $Q$ estimation, the CRLB in Eq.~\eqref{eq:crlb-Q-final} exhibits different behaviors:

\begin{itemize}
\item \textbf{Low-$Q$ limit} ($Q \ll 1$): With $1 + 4Q^2 \approx 1$,
  \begin{equation}
  \sigma_Q^2 \ge \frac{\sigma^2 \tau^2}{4 A_0^2 \Delta S_2}.
  \end{equation}
  For $T \gg \tau$, using $\Delta S_2 \approx \tau^3/(8T_s)$,
  \begin{equation}
  \sigma_Q^2 \ge \frac{2T_s \sigma^2}{A_0^2 \tau}.
  \end{equation}

\item \textbf{High-$Q$ limit} ($Q \gg 1$): With $1 + 4Q^2 \approx 4Q^2$,
  \begin{equation}
  \sigma_Q^2 \ge \frac{\sigma^2 \tau^2 Q^2}{A_0^2 \Delta S_2},
  \quad\Rightarrow\quad
  \frac{\sigma_Q}{Q} \ge \frac{\sigma \tau}{A_0 \sqrt{\Delta S_2}}.
  \end{equation}
  For $T \gg \tau$ (rapid-decay regime), using $\Delta S_2 \approx \tau^3/(8T_s)$, the relative error scales as $\sigma_Q/Q \propto 1/\sqrt{\tau} \propto 1/\sqrt{Q}$, as shown in Fig.~\ref{fig:crlb-q-vs-q}. However, for very large $Q$, the decay time $\tau = Q/(\pi f_0)$ becomes long compared to the observation time $T$, and the system transitions to the slow-decay regime ($T \ll \tau$). In this regime, using $\Delta S_2 \approx NT^2/12$ and $\tau = Q/(\pi f_0)$, the relative error degrades as $\sigma_Q/Q \propto Q/(\sqrt{N}T)$, worsening with increasing $Q$. This transition explains the precision degradation observed at very large $Q$ values in Fig.~\ref{fig:crlb-q-vs-q}, where the observation time becomes insufficient to capture the full decay.
\end{itemize}

The dependence on observation time $T/\tau$ is shown in Fig.~\ref{fig:crlb-q-tau-ratio}.

\begin{figure}[htbp]
\centering
\includegraphics{crlb_freq_vs_tau_ratio.pdf}
\caption{Frequency CRLB as a function of observation time ratio $T/\tau$. The transition from slow-decay ($T \ll \tau$) to rapid-decay ($T \gg \tau$) regimes is evident, with the approximated asymptotic limits shown as dashed lines. Parameters: $f_0 = \SI{5}{Hz}$, $f_s = \SI{100}{Hz}$, $\rho_0 = \SI{60}{dB}$, $\tau = \SI{100}{s}$.}
\label{fig:crlb-freq-tau-ratio}
\end{figure}

\begin{figure}[htbp]
\centering
\includegraphics{crlb_freq_vs_snr.pdf}
\caption{Frequency CRLB as a function of initial SNR $\rho_0$. The $1/\sqrt{\rho_0}$ scaling is shown as a dashed line. Parameters: $f_0 = \SI{5}{Hz}$, $f_s = \SI{100}{Hz}$, $N = 10^5$ samples, $\tau = \SI{100}{s}$.}
\label{fig:crlb-freq-snr}
\end{figure}

\begin{figure}[htbp]
\centering
\includegraphics{crlb_q_vs_q.pdf}
\caption{Q-factor CRLB relative error $\sigma_Q/Q$ as a function of $Q$. The high-$Q$ scaling $\propto 1/\sqrt{Q}$ (valid for $T \gg \tau$, rapid-decay regime) is shown as a dashed line. At very large $Q$, precision degrades because $\tau = Q/(\pi f_0)$ becomes long compared to the observation time $T$, transitioning to the slow-decay regime ($T \ll \tau$) where the relative error scales as $\propto Q$ rather than $\propto 1/\sqrt{Q}$. This occurs when the observation window is insufficient to capture the full exponential decay. Parameters: $f_0 = \SI{5}{Hz}$, $f_s = \SI{100}{Hz}$, $N = 10^5$ samples, $\rho_0 = \SI{60}{dB}$.}
\label{fig:crlb-q-vs-q}
\end{figure}

\begin{figure}[htbp]
\centering
\includegraphics{crlb_q_vs_tau_ratio.pdf}
\caption{Q-factor CRLB relative error $\sigma_Q/Q$ as a function of observation time ratio $T/\tau$. Parameters: $f_0 = \SI{5}{Hz}$, $f_s = \SI{100}{Hz}$, $Q = 10^4$, $\rho_0 = \SI{60}{dB}$.}
\label{fig:crlb-q-tau-ratio}
\end{figure}

\section{Nonlinear least squares (NLS)}

Directly fitting the parametric ring-down model
\begin{equation}
x_n = A_0 e^{-t_n/\tau}\cos(2\pi f\,t_n + \phi) + c + w_n
\end{equation}
by minimizing the weighted least-squares cost
\begin{equation}
J = \sum_{n=0}^{N-1} \left[x_n - A_0 e^{-t_n/\tau}\cos(2\pi f t_n + \phi) - c\right]^2
\end{equation}
over parameters $(A_0,\phi,f,\tau,c)$ provides a maximum-likelihood estimator under Gaussian noise assumptions.

\section{Frequency-domain methods (DFT)}

\subsection{DFT peak location}

The windowed discrete Fourier transform is
\begin{equation}
X[k] = \sum_{n=0}^{N-1} w_n x_n\,e^{-i 2\pi kn/N},\qquad k=0,1,\dots,N-1,
\end{equation}
where $w_n$ is a window function. We use a rectangular window (i.e., no window, $w_n = 1$), which provides better results (lower variance) for ring-down signals due to more efficient use of the data. The frequency grid spacing is
\begin{equation}
\Delta f = \frac{f_s}{N} = \frac{1}{T}.
\end{equation}

For a ring-down signal, the exponential amplitude modulation broadens the spectral peak. The Fourier transform of $A_0 e^{-t/\tau}\cos(2\pi f_0 t)$ has a Lorentzian shape centered at $f_0$ with width approximately $1/(2\pi\tau)$.

\subsection{Peak fitting with Lorentzian function}

Since the Fourier transform of ring-down signals has a Lorentzian shape, fitting a Lorentzian function to the power spectrum is more appropriate than, e.g., a parabolic interpolation. A practical estimator:

\begin{enumerate}
\item Find the peak bin $k_{\max} = \arg\max_k |X[k]|^2$.
\item Fit a Lorentzian function to the power spectrum $P[k] = |X[k]|^2$ around the peak using multiple bins (e.g., 5--10 bins).
\item Extract the center frequency $f_0$ from the fitted Lorentzian parameters.
\end{enumerate}

The Lorentzian function is parameterized as:
\begin{equation}
P(f) = \frac{A}{(f - f_0)^2 + (\gamma/2)^2} + P_{\text{offset}},
\end{equation}
where $A$ is the amplitude, $f_0$ is the center frequency, $\gamma$ is the full width at half maximum (FWHM), and $P_{\text{offset}}$ is a background offset. The fitting is performed using nonlinear least squares on the power spectrum values around the peak bin.

\subsection{NLS vs.\ DFT efficiency}
\label{sec:dft-efficiency}

The CRLB in Eq.~\eqref{eq:crlb-ringdown} applies to any unbiased estimator, but DFT-based methods will generally not achieve it. The inefficiency arises from information loss:
 
\begin{itemize} 
\item {Windowing}: Applying a window function leads to information loss; this is why we use a rectangular window.
\item {Phase discarding}: Using only the power spectrum discards half the information in the complex DFT coefficients. 
\item {Discretization bias}: The DFT samples the spectrum at discrete frequencies; interpolation (even with Lorentzian fitting) cannot fully recover the continuous-parameter information. 
\end{itemize} 

The above limitations ensure that DFT methods remain statistically suboptimal relative to NLS for ring-down signals.

\section{Numerical analysis}

We provide a Python script that implements NLS and DFT methods for ring-down signals and performs Monte Carlo simulations. The script generates a time series sampled at $f_s = \SI{100}{Hz}$ with $N = 10^6$ samples (observation time $T = \SI{10000}{s}$) containing a ring-down signal at $f_0 = \SI{5}{Hz}$ with quality factor $Q$ and additive white Gaussian noise.

For each Monte Carlo trial, the script:
\begin{enumerate}
\item Generates an independent realization with random noise and initial phase $\phi_0 \sim \mathcal{U}(-\pi, \pi)$.
\item Calculates the CRLB from the explicit Fisher information matrix derivation for both frequency and quality factor $Q$.
\item Estimates frequency using nonlinear least squares and windowed DFT peak fitting with Lorentzian function.
\item Estimates quality factor $Q$ using NLS (by estimating both frequency and decay time $\tau$, then computing $Q = \pi f \tau$).
\item Computes error statistics: mean bias, standard deviation, and root-mean-square error for both frequency and $Q$ estimates.
\end{enumerate}

The script produces figures comparing the performance of the estimation methods for ring-down signals. Frequency estimation results are shown in Figs.~\ref{fig:individual-v6}--\ref{fig:performance-v6}, and quality factor estimation results are shown in Figs.~\ref{fig:q-individual-v6}--\ref{fig:q-performance-v6}.

\begin{figure}[htbp]
\centering
\includegraphics{freq_estimation_ringdown_v6_individual.pdf}
\caption{Error distributions for NLS and DFT frequency estimation methods applied to ring-down signals. Each panel shows the histogram of frequency estimation errors from 100 Monte Carlo trials. The vertical dashed lines indicate the CRLB standard deviation calculated from the explicit Fisher information matrix. Parameters: $f_0 = \SI{5}{Hz}$, $f_s = \SI{100}{Hz}$, $N = 10^6$ samples ($T = \SI{10000}{s}$), initial $\text{SNR} = \SI{60}{dB}$, quality factor $Q = 10^4$ (decay time $\tau = \SI{636.6}{s}$). The NLS method (top) achieves standard deviations close to the CRLB, while the DFT method with Lorentzian fitting (bottom) shows improved accuracy compared to parabolic interpolation due to better matching of the spectral peak shape.}
\label{fig:individual-v6}
\end{figure}

\begin{figure}[htbp]
\centering
\includegraphics{freq_estimation_ringdown_v6_aggregate.pdf}
\caption{Direct comparison of NLS and DFT estimation methods for ring-down signals. Left panel: Overlaid histograms showing the error distributions. Right panel: Box plots comparing the error distributions. The horizontal dashed lines indicate zero error and the CRLB. The NLS method's explicit ring-down model provides efficient estimation, while the DFT method yields higher uncertainty.}
\label{fig:aggregate-v6}
\end{figure}

\begin{figure}[htbp]
\centering
\includegraphics{freq_estimation_ringdown_v6_performance.pdf}
\caption{Performance metrics comparison for ring-down signals. Left panel: Standard deviation of frequency estimation errors for each method, plotted on a logarithmic scale. The horizontal dashed line indicates the CRLB. Right panel: Statistical efficiency $\eta = \sigma_{\text{CRLB}}/\sigma_{\text{method}}$, where $\eta = 1$ corresponds to achieving the CRLB. The NLS method achieves efficiency near unity, demonstrating its statistical optimality for ring-down signals, while the DFT method with Lorentzian fitting achieves improved efficiency compared to parabolic interpolation due to reduced bias from better matching the spectral peak shape.}
\label{fig:performance-v6}
\end{figure}

\subsection{Quality factor estimation}

The quality factor $Q$ is estimated from the NLS method by jointly estimating frequency $f$ and decay time $\tau$, then computing $Q = \pi f \tau$. The CRLB for $Q$ estimation is calculated from the explicit Fisher information matrix derivation in Sec.~\ref{sec:crlb-scaling}, accounting for the covariance between frequency and decay time estimates (see Eq.~\eqref{eq:crlb-Q-final}).

\begin{figure}[htbp]
\centering
\includegraphics{q_estimation_ringdown_v6_individual.pdf}
\caption{Error distribution for NLS quality factor ($Q$) estimation method applied to ring-down signals. The histogram shows the distribution of $Q$ estimation errors from 100 Monte Carlo trials. The vertical dashed lines indicate the CRLB standard deviation calculated from the explicit Fisher information matrix. Parameters: $f_0 = \SI{5}{Hz}$, $f_s = \SI{100}{Hz}$, $N = 10^6$ samples ($T = \SI{10000}{s}$), initial $\text{SNR} = \SI{60}{dB}$, quality factor $Q = 10^4$ (decay time $\tau = \SI{636.6}{s}$). The NLS method achieves standard deviations close to the Q-CRLB, demonstrating efficient estimation of the quality factor.}
\label{fig:q-individual-v6}
\end{figure}

\begin{figure}[htbp]
\centering
\includegraphics{q_estimation_ringdown_v6_performance.pdf}
\caption{Performance metrics for quality factor ($Q$) estimation. Left panel: Standard deviation of $Q$ estimation errors for the NLS method, plotted on a logarithmic scale. The horizontal dashed line indicates the Q-CRLB. Right panel: Statistical efficiency $\eta = \sigma_{\text{CRLB}}/\sigma_{\text{method}}$, where $\eta = 1$ corresponds to achieving the CRLB. The NLS method achieves efficiency near unity for $Q$ estimation, demonstrating that the joint estimation of frequency and decay time provides statistically efficient quality factor estimates.}
\label{fig:q-performance-v6}
\end{figure}

\section{Comparison and conclusions}

Nonlinear least squares with explicit ring-down model provides statistically efficient frequency estimation, achieving the CRLB derived from the Fisher information matrix. The DFT peak fitting method with Lorentzian function and rectangular window is computationally efficient, but suffers from statistical inefficiency due to information loss.

The exponential decay reduces the effective observation time and SNR, degrading estimation performance compared to constant-amplitude signals. The degradation depends on the ratio $T/\tau$: for $T \ll \tau$, performance is similar to a constant-amplitude signal, while for $T \gtrsim \tau$, later samples contribute little information due to reduced amplitude.

\begin{thebibliography}{9}
\bibitem{kay}
S.~M. Kay, \textit{Fundamentals of Statistical Signal Processing: Estimation Theory}. Prentice Hall, 1993.
\bibitem{rife}
D.~C. Rife and R.~R. Boorstyn, ``Single tone parameter estimation from discrete-time observations,'' \textit{IEEE Trans.\ Information Theory}, 1974.
\end{thebibliography}

\end{document}

